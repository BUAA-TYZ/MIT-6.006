%
% 6.006 problem set 0 solutions template
%
\documentclass[12pt,twoside]{article}

\input{macros-sp20}
\newcommand{\theproblemsetnum}{0}

\title{6.006 Problem Set 0}

\begin{document}

\handout{Problem Set \theproblemsetnum}

\setlength{\parindent}{0pt}
\medskip\hrulefill\medskip

{\bf Name:} BUAA-TYZ 

\medskip\hrulefill

%%%%%%%%%%%%%%%%%%%%%%%%%%%%%%%%%%%%%%%%%%%%%%%%%%%%%
% See below for common and useful latex constructs. %
%%%%%%%%%%%%%%%%%%%%%%%%%%%%%%%%%%%%%%%%%%%%%%%%%%%%%

% Some useful commands:
% $f(x) = \Theta(x)$
% $T(x, y) \leq \log(x) + 2^y + \binom{2n}{n}$
% \ttt{code\_function}


% You can create unnumbered lists as follows:
% \begin{itemize}
%     \item First item in a list
%         \begin{itemize}
%             \item First item in a list
%                 \begin{itemize}
%                     \item First item in a list
%                     \item Second item in a list
%                 \end{itemize}
%             \item Second item in a list
%         \end{itemize}
%     \item Second item in a list
% \end{itemize}

% You can create numbered lists as follows:
% \begin{enumerate}
%     \item First item in a list
%     \item Second item in a list
%     \item Third item in a list
% \end{enumerate}

% You can write aligned equations as follows:
% \begin{align}
%     \begin{split}
%         (x+y)^3 &= (x+y)^2(x+y) \\
%                 &= (x^2+2xy+y^2)(x+y) \\
%                 &= (x^3+2x^2y+xy^2) + (x^2y+2xy^2+y^3) \\
%                 &= x^3+3x^2y+3xy^2+y^3
%     \end{split}
% \end{align}

% You can create grids/matrices as follows:
% \begin{align}
%     A =
%     \begin{bmatrix}
%         A_{11} & A_{21} \\
%         A_{21} & A_{22}
%     \end{bmatrix}
% \end{align}

\begin{problems}

\problem  % Problem 1

Known: $A=\{i+\binom{5}{i}, i\in \mathbb{Z} \; and \;0 \le i \ge 4\} and $
$B=\{3i\;|\;i \; \in \; \left\{1,2,4,5\right\}\}$

Enumerating i, we include that $A=\{1,6,12,13,9\} \; and \;B=\{3,6,12,15\}$
\begin{problemparts}
\problempart % Problem 1a
$A \cap B=\{6,12\}$
\problempart % Problem 1b
$A \cup B=\{1,3,6,9,12,13,15\} \Rightarrow |A \cup B|=7$
\problempart % Problem 1c
$A-B=\{1,9,13\}\Rightarrow|A-B|=3$
\end{problemparts}

\problem  % Problem 2
Known: X be the random variable representing the number of heads seen after flipping 
a fair coin three times. Let Y be the random variable representing the outcome of rolling two fair 
six-sided dice and multiplying their values.
\begin{problemparts}
\problempart % Problem 2a
The possible value of X is $\{0,1,2,3\}$

$E[X]=\sum_{i=0}^{3}xp(x)=0*\frac{1}{8}+1*\frac{3}{8}+2*\frac{3}{8}+3*\frac{1}{8}=1.5$
\problempart % Problem 2b
The possible value of Y is $\{1,2,3,4,5,6,8,9,10,12,15,16,18,20,24,25,30,36\}$

$E[Y]=...=12.25$
\problempart % Problem 2c
$E[X+Y]=E[X]+E[Y]=13.75$
\end{problemparts}

\problem  % Problem 3
Known: $A=6000/6=100\quad and \quad B=60\;mod\;42=18$
\begin{problemparts}
\problempart % Problem 3a
True
\problempart % Problem 3b
False
\problempart % Problem 3c
False
\end{problemparts}

\problem  % Problem 4
Prove by induction that $\sum_{i=1}^{n}i^3=\left[\frac{n(n+1)}{2}\right]^2, \forall \;n\ge1$

\begin{itemize}
    \item \;Base case: $n=1:\;$ Obviously.
    \item \;Suppose $n=k:\;$ Assume the formula is right. Then for $n=k+1$, we have:
    $$
    \sum_{i=1}^{k+1}i^3=\sum_{i=1}^{k}i^3+(k+1)^3=\left[\frac{k(k+1)}{2}\right]^2+(k+1)^3
    =(k+1)^2*(\frac{k^2}{4}+k+1)=\left[\frac{(k+1)(k+2)}{2}\right]^2
    $$
    \item \;Conclusion: $\sum_{i=1}^{n}i^3=\left[\frac{n(n+1)}{2}\right]^2, \forall \;n\ge1$
\end{itemize}

\newpage
\problem  % Problem 5
Prove by induction that every connected undirected graph $G = (V, E)$ for which 
$|E| = |V| - 1$ is acyclic. 

\begin{itemize}
    \item \;
    Base case: $|E|=1:$ Two nodes and one edge, there is no cycle.
    \item \;
    Suppose that $|E|=k:$ Assume this is right. Then we add one node and one edge, so 
    this node can only be linked to one another node, which can't bring a cycle.
    \item \;
    Conclusion: Every connected undirected graph $G = (V, E)$ for which 
    $|E| = |V| - 1$ is acyclic. 
\end{itemize}


\problem  % Problem 6
Submit your implementation to {\small\url{alg.mit.edu}}.

\begin{lstlisting}
def count_long_subarray(A):
    '''
    Input:  A     | Python Tuple of positive integers
    Output: count | number of longest increasing subarrays of A
    '''
    max_length_count = 0
    ##################
    # YOUR CODE HERE #
    ##################
    local_max_length = 1
    global_max_length = 1
    for i in range(1, length):
        if A[i] > A[i - 1]:
            local_max_length += 1
        else:
            local_max_length = 1
        if local_max_length == global_max_length:
            max_length_count += 1
        elif local_max_length > global_max_length:
            global_max_length = local_max_length
            max_length_count = 1
    return max_length_count
\end{lstlisting}

\end{problems}

\end{document}
