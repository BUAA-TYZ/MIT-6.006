%
% 6.006 problem set 2 solutions template
%
\documentclass[12pt,twoside]{article}

\input{macros-sp20}
\newcommand{\theproblemsetnum}{2}

\title{6.006 Problem Set 2}

\begin{document}

\handout{Problem Set \theproblemsetnum}

\setlength{\parindent}{0pt}
\medskip\hrulefill\medskip

{\bf Name:} Your Name

\medskip

{\bf Collaborators:} Name1, Name2

\medskip\hrulefill

%%%%%%%%%%%%%%%%%%%%%%%%%%%%%%%%%%%%%%%%%%%%%%%%%%%%%
% See below for common and useful latex constructs. %
%%%%%%%%%%%%%%%%%%%%%%%%%%%%%%%%%%%%%%%%%%%%%%%%%%%%%

% Some useful commands:
%$f(x) = \Theta(x)$
%$T(x, y) \leq \log(x) + 2^y + \binom{2n}{n}$
% {\tt code\_function}


% You can create unnumbered lists as follows:
%\begin{itemize}
%    \item First item in a list
%        \begin{itemize}
%            \item First item in a list
%                \begin{itemize}
%                    \item First item in a list
%                    \item Second item in a list
%                \end{itemize}
%            \item Second item in a list
%        \end{itemize}
%    \item Second item in a list
%\end{itemize}

% You can create numbered lists as follows:
%\begin{enumerate}
%    \item First item in a list
%    \item Second item in a list
%    \item Third item in a list
%\end{enumerate}

% You can write aligned equations as follows:
%\begin{align}
%    \begin{split}
%        (x+y)^3 &= (x+y)^2(x+y) \\
%                &= (x^2+2xy+y^2)(x+y) \\
%                &= (x^3+2x^2y+xy^2) + (x^2y+2xy^2+y^3) \\
%                &= x^3+3x^2y+3xy^2+y^3
%    \end{split}
%\end{align}

% You can create grids/matrices as follows:
%\begin{align}
%    A =
%    \begin{bmatrix}
%        A_{11} & A_{21} \\
%        A_{21} & A_{22}
%    \end{bmatrix}
%\end{align}

% You can include images and PDFs as follows:
% \includegraphics[width=0.5\textwidth]{img.jpg}

\begin{problems}

\problem  % Problem 1

\begin{problemparts}
\problempart % Problem 1a
% An example of embedding images, in case you want to include a drawing of a tree.
% \begin{center}
%   \includegraphics[width=0.5\textwidth]{img.jpg}
% \end{center}
$T(n)=4T(\frac{n}{2})+O(n)$

% So clever! Use node and $\;$ to show the empty!
\begin{center}
\Tree
[.\node[draw]{$cn$}; 
[.\node[draw]{$ $}; 
[.\node[draw]{$\,$};]
[.\node[draw]{$\,$};]
[.\node[draw]{$\,$};]
[.\node[draw]{$\,$};]
] 
[.\node[draw]{$\,$}; 
[.\node[draw]{$\,$};]
[.\node[draw]{$\,$};]
[.\node[draw]{$\,$};]
[.\node[draw]{$\,$};]
] 
[.\node[draw]{$\,$}; 
[.\node[draw]{$\,$};]
[.\node[draw]{$\,$};]
[.\node[draw]{$\,$};]
[.\node[draw]{$\,$};]
] 
[.\node[draw]{$c\frac{n}{2^i}$}; 
[.\node[draw]{$\,$};] 
[.\node[draw]{$\,$};] 
[.\node[draw]{$\,$};] 
[.\node[draw]{$c$};] 
] 
]
\end{center}
$T(n)=\sum_{i=0}^{logn}4^i\frac{n}{2^i}=n\sum_{i=0}^{logn}2^i=n(2n-1)=\Theta(n^2)$

\problempart % Problem 1b
\begin{align}   
    T(n) &= 3T(\frac{n}{\sqrt{2}}) + O(n^4)=\sum_{i=0}^{\log_{\sqrt{2}}n}3^i(\frac{n^4}{4^i}) \\
    &=4n^4(1-(\frac{3}{4})^{log_{\sqrt{2}}n+1})=O(n^4)
\end{align}


\problempart % Problem 1c
\begin{align}
    T(n) &= 2T(\frac{n}{2})+5nlogn =\sum_{i=0}^{logn}2^i \frac{5nlogn}{2^i}  \\
    &=5nlogn(logn+1)=O(nlog^{2}n)
\end{align}

\end{problemparts}

\newpage
\problem  % Problem 2

\begin{problemparts}
\problempart % Problem 2a
\problempart % Problem 2b
\problempart % Problem 2c
\end{problemparts}

\newpage
\problem  % Problem 3

\newpage
\problem  % Problem 4

\newpage
\problem  % Problem 5

\begin{problemparts}
\problempart % Problem 5a
\problempart % Problem 5b
\problempart Submit your implementation to {\small\url{alg.mit.edu}}.
\end{problemparts}

\end{problems}

\end{document}
